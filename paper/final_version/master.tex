\documentclass[11pt, notitlepage]{report}
%^ hack for subfiles

\usepackage{enumerate}
\usepackage[OT1]{fontenc}
\usepackage{amsmath,amssymb}
\usepackage[numbers]{natbib}
\usepackage[usenames]{color}
%\usepackage[dvips]{graphicx}
%\usepackage{bm}

\usepackage{smile}
\usepackage{multirow}
\usepackage{listings}
% \usepackage{geometry}
\usepackage{pdflscape}
%\usepackage{rotating}
\usepackage{todonotes}
\usepackage{tikz}
\usetikzlibrary{shapes, arrows, fit, calc}
\usepackage{array}% http://ctan.org/pkg/array
\newcolumntype{M}{>{\centering\arraybackslash}m{}}


\usepackage{silence}% Filter out unwanted warnings and error messages
\WarningFilter{pdftex}{destination with the same}
%\usepackage{subfigure}
%\usepackage{makecell}
\usepackage[colorlinks = false,
            % linkcolor=red,
            % anchorcolor=blue,
            % citecolor=blue
            ]{hyperref}
            
\graphicspath{ {/Users/daway/Documents/Princeton/Thesis2017/plots/} }
\setlength{\oddsidemargin}{0in}
\setlength{\evensidemargin}{0in}
\setlength{\textheight}{9in}
\setlength{\textwidth}{6.5in}
\setlength{\topmargin}{-0.5in}

% \usepackage{scrextend}
\usepackage{float}
\usepackage{enumerate}
\usepackage{amsfonts}
\usepackage{amssymb}
\usepackage{afterpage}
\usepackage{relsize}
\usepackage{tabularx}
\newcolumntype{Y}{>{\RaggedRight\arraybackslash}X} 

%\mathbb{set}

% Sample macros -- how you define new commands
% My own set of frequently-used macros have grown to many hundreds of lines.
% Here are some simple samples.

\newcommand{\Adv}{{\mathbf{Adv}}}       
\newcommand{\prp}{{\mathrm{prp}}}                  % How to define new commands 
\newcommand{\calK}{{\cal K}}
\newcommand{\outputs}{{\Rightarrow}}                
\newcommand{\getsr}{{\:\stackrel{{\scriptscriptstyle\hspace{0.2em}\$}}{\leftarrow}\:}}
\newcommand{\andthen}{{\::\;\;}}    % \, \: \; for thinspace, medspace, thickspace
\newcommand{\Rand}[1]{{\mathrm{Rand}[{#1}]}}       % A command with one argument
\newcommand{\Perm}[1]{{\mathrm{Perm}[{#1}]}}       
\newcommand{\Randd}[2]{{\mathrm{Rand}[{#1},{#2}]}} % and with two arguments
\newcommand{\thetahat}{{\hat\theta}}
\providecommand{\floor}[1]{\left \lfloor #1 \right \rfloor }
\usepackage{mathtools}
\DeclarePairedDelimiter{\ceil}{\lceil}{\rceil}

\newcommand*\justify{%
	\fontdimen2\font=0.4em% interword space
	\fontdimen3\font=0.2em% interword stretch
	\fontdimen4\font=0.1em% interword shrink
	\fontdimen7\font=0.1em% extra space
	\hyphenchar\font=`\-% allowing hyphenation
}

\newlength\tindent
\setlength{\tindent}{\parindent}
\setlength{\parindent}{0pt}
\renewcommand{\indent}{\hspace*{\tindent}}

%\renewcommand*{\chapterheadstartvskip}{\vspace*{1cm}}
%\renewcommand*{\chapterheadendvskip}{\vspace{2cm}}

\usepackage{mathrsfs}

\usepackage{fullpage}
\renewcommand{\baselinestretch}{1.0}


\usepackage{hyperref}
%\usepackage[protrusion=false,expansion=true]{microtype}
%\usepackage[activate={true,nocompatibility},final,tracking=true,kerning=true,spacing=true,factor=1100,stretch=10,shrink=10]{microtype}
\usepackage[english]{babel}
\usepackage[    protrusion=true,
            expansion=true,
            final,
            babel
                ]{microtype}
\usepackage{setspace}
\usepackage{enumitem}

% \usepackage{parskip}
\lstset{frame=tb,
  language=Python,
  aboveskip=3mm,
  belowskip=3mm,
  showstringspaces=false,
  columns=flexible,
  basicstyle={\small\ttfamily},
  numbers=none,
  numberstyle=\tiny\color{gray},
  keywordstyle=\color{blue},
  commentstyle=\color{dkgreen},
  stringstyle=\color{mauve},
  breaklines=true,
  breakatwhitespace=true
  tabsize=3
}

%% Thesis requirements for binding
\usepackage[top=1in, bottom=1in, left=1.5in, right=1in]{geometry}

%% modular files
\usepackage{subfiles}

\setlength\parskip{\baselineskip}

%% chapter-fixing
\newcommand{\mychapter}[2]{
    \setcounter{chapter}{#1}
%    \setcounter{section}{0}
    \chapter{#2}
    \addcontentsline{toc}{chapter}{#2}
}

%% landscaping
%\usepackage[toc,page]{appendix}
\usepackage[titletoc]{appendix}
\usepackage{rotating}
\usepackage{pdflscape}
\usepackage{textcomp} % https://tex.stackexchange.com/questions/165115/getting-not-defining-perthousnad-and-not-defining-micro-when-compiling-beamer
\usepackage{gensymb}

%% indicator function
\usepackage{bbm}

%% make C++ look pretty
\newcommand{\CPP}
{C\nolinebreak[4]\hspace{-.05em}\raisebox{.22ex}{\footnotesize\bf ++}\hspace{.3em}}

%% reset sub-counts for figures within chapters
\usepackage{chngcntr}
% \counterwithout{figure}{chapter}
\counterwithin{figure}{chapter}

\begin{document}
\hypersetup{pageanchor=false}
\pagenumbering{Alph}
\begin{titlepage}

\newcommand{\HRule}{\rule{.8\linewidth}{0.5mm}} % Defines a new command for the horizontal lines, change thickness here
\renewenvironment{abstract}
 {\small
  \begin{center}
  \bfseries \abstractname\vspace{-.5em}\vspace{0pt}
  \end{center}
  \list{}{
    \setlength{\leftmargin}{.5cm}%
    \setlength{\rightmargin}{\leftmargin}%
  }%
  \item\relax}
 {\endlist}


\center % Center everything on the page

%----------------------------------------------------------------------------------------
%   HEADING SECTIONS
%----------------------------------------------------------------------------------------

\textsc{\LARGE Princeton University}\\[0.5cm] % Name of your university/college
\textsc{\large Computer Science}\\[0.5cm] % Major heading such as course name

\vspace{2em}
%----------------------------------------------------------------------------------------
%   TITLE SECTION
%----------------------------------------------------------------------------------------

\HRule \\[0.2cm]
%{\Large \bfseries Weakly Learning\\
%	a Fine-Grained Image Similarity\\
{\LARGE \bfseries Learning Fine-Grained\\
	Image Similarity Through\\
	Weak Supervision\\
\par}
\HRule \\[1.5cm]
 
\vspace{2em}
%----------------------------------------------------------------------------------------
%   AUTHOR SECTION
%----------------------------------------------------------------------------------------

\begin{minipage}{0.4\textwidth}
\begin{flushleft} \large
\emph{Author:}\\
Daway \textsc{Chou-Ren} % Your name
\end{flushleft}
\end{minipage}
~
\begin{minipage}{0.4\textwidth}
\begin{flushright} \large
\emph{Advisor:} \\
Dr. Szymon \textsc{Rusinkcewicz} % Supervisor's Name
\end{flushright}
\end{minipage}\\[1cm]

\vspace{2em}
%----------------------------------------------------------------------------------------
%   LOGO SECTION
%----------------------------------------------------------------------------------------
\begin{figure}[H]
\centering
\includegraphics[width=30mm]{figs/princeton_shield}
\end{figure}

\vspace{4em}
 
%----------------------------------------------------------------------------------------
\textsc{
Submitted in partial fulfillment\\
of the requirements for the degree of\\
Bachelor of Science in Engineering\\
Department of Computer Science\\
Princeton University
}\\[0.5cm]

\vspace{2em}
%----------------------------------------------------------------------------------------
%   DATE SECTION
%----------------------------------------------------------------------------------------

\textsc{\large \today}\\[5mm] % Date, change the \today to a set date if you want to be precise
\pagestyle{empty}

%----------------------------------------------------------------------------------------
% Statements
%----------------------------------------------------------------------------------------

\begin{minipage}{\textwidth}
\begin{flushleft} \large
\textsc{I hereby declare that I am the sole author of this thesis.}\par
\vspace{1em}
\textsc{I authorize Princeton University to lend this thesis to other institutions or individuals for
the purpose of scholarly research.}
\vspace{6em}
\end{flushleft}
\begin{flushright}
\noindent\begin{tabular}{ll}
\makebox[2.5in]{\hrulefill}{\hrulefill}\\
\textsc{Daway Chou-Ren}\\
\end{tabular}
\end{flushright}
\vspace{6em}
\begin{flushleft} \large
\textsc{I further authorize Princeton University to reproduce this thesis by photocopying or by
other means, in total or in part, at the request of other institutions or individuals for the
purpose of scholarly research.}
\vspace{6em}
\end{flushleft}
\begin{flushright}
\noindent\begin{tabular}{ll}
\makebox[2.5in]{\hrulefill}{\hrulefill}\\
\textsc{Daway Chou-Ren}\\
\end{tabular}
\end{flushright}
\end{minipage}

\pagestyle{empty}
\newpage
%----------------------------------------------------------------------------------------
% Acknowledgements
%----------------------------------------------------------------------------------------
\begin{minipage}{\textwidth}
{ \LARGE \bfseries Acknowledgements\par}
\parskip=\baselineskip \advance\parskip by 0pt plus 2pt
\end{minipage}

\end{titlepage}

\hypersetup{pageanchor=false}
\tableofcontents

\hypersetup{pageanchor=true}
\thispagestyle{empty}
\cleardoublepage
% \pagestyle{headings}
\pagenumbering{arabic}

\chapter*{Abstract} \label{section:abstract}
	\addcontentsline{toc}{chapter}{Abstract}
	Learning an embedding useful for fine-grained image similarity on a human perception level requires detecting three general classes of attributes: 1) high level semantic characteristics, such as emotional qualities or aesthetic styles; 2) complex visual descriptors such as the objects that comprise the focus on an image; and 3) low level descriptors such as textures and colors which can influence human perception of the first two classes. In this paper, we present a deep multi-module Siamese network that allows for the learning and blending of these three attribute classes. We train the network using weak supervision by sampling publicly available Flickr images for image pairs deemed similar and dissimilar by a set of constructed heuristics. Experiments show that this weakly supervised model learns a highly generalizable similarity embedding which surpasses the effectiveness of ImageNet trained models.
	

\chapter{Introduction} \label{chapter:intro}
\subfile{Introduction}

\chapter{Background} \label{chapter:background}
\subfile{Background}

\chapter{Network Architecture and Training} \label{chapter:network}
\subfile{Network}

\chapter{Data} \label{chapter:data}
\subfile{Data}

\chapter{Experiments and Discussion} \label{chapter:experiments}
\subfile{Experiments}

\chapter{Conclusion} \label{chapter:conclusion}
\subfile{Conclusion}

\begin{appendices}
%\appendix
\subfile{Appendix}
\end{appendices}

\setlength{\bibsep}{1pt}
{
\bibliographystyle{IEEEtranS}
\bibliography{master}
}
\end{document} 
